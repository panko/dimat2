\documentclass[12pt]{article}

\usepackage[utf8]{inputenc}
\usepackage{xcolor}   %May be necessary if you want to color links
\usepackage{hyperref}
\usepackage{amsmath}
\hypersetup{
    colorlinks=false, %set true if you want colored links
    linkbordercolor = {white}
}
 
\title{\Huge Kalkulus 1 \\
\LARGE University of Debrecen \\
\large DE-IK P.T.I.}
\author{Peter Panko}

 
\begin{document}

\clearpage\maketitle
\thispagestyle{empty}
\newpage


\tableofcontents 

\newpage



%%%%%%%%%%%%%%%%%%
%%%%%%BEUGRÓ%%%%%%
%%%%%%%%%%%%%%%%%%

\section{Beugró}

%%%%%%%%%%%%%%%%%%%
%%%%%ALAPDEFEK%%%%%
%%%%%%%%%%%%%%%%%%%

\subsection{Alapdefiníciók}

\subsubsection{Halmaz unió}

\begin{center}
$A \cup B = \{x \in X\ |\ x \in A\ \text{vagy}\ x \in B\}$
\end{center}
halmazt értjük.

\subsubsection{Halmaz metszet}

Legyen $X$ egy nemüres halmaz, $A, B \subset X$. Ekkor az $A$ és $B$ halmazok metszetén az
\begin{center}
$A \cap B = \{x \in X\ |\ x \in A\ \text{és}\ x \in B\}$
\end{center}
halmazt értjük.




\subsubsection{Felülről korlátos halmaz}

Legyen (A, ≤) egy parciálisan rendezett halmaz. Azt mondjuk, hogy a B ⊂ A halmaz
felülr ˝ol korlátos, ha létezik olyan a ∈ A, hogy tetsz ˝oleges b ∈ B esetén b ≤ a teljesül. Ekkor azt mondjuk,
hogy az a elem a B halmaz fels ˝o korlátja.

\subsubsection{Alulról korlátos halmaz}

Legyen (A, ≤) egy parciálisan rendezett halmaz. Azt mondjuk, hogy a B ⊂ A halmaz
alulról korlátos, ha létezik olyan a ∈ A, hogy tetsz ˝oleges b ∈ B esetén a ≤ b teljesül. Ekkor azt mondjuk,
hogy az a elem a B halmaz alsó korlátja.
A B halmaz pontos fels ˝o korlátja – ha létezik, akkor – egyértelm ˝u.

\subsubsection{pontos felső korlát}

Legyen (A, ≤) egy parciálisan rendezett halmaz és B ⊂ A egy felülr ˝ol korlátos halmaz.
Ekkor a B halmaz fels ˝o korlátainak a minimumát a B halmaz pontos fels ˝o korlátjának hívjuk. Erre a
sup B jelölést használjuk.

\subsubsection{pontos alsó korlát}

Legyen (A, ≤) egy parciálisan rendezett halmaz és B ⊂ A egy alulról korlátos halmaz.
Ekkor a B halmaz alsó korlátainak a maximumát a B halmaz pontos alsó korlátjának hívjuk. Erre az
inf B jelölést használjuk.
1.2.9. Megjegyzés. A B halmaz pontos alsó korlátja – ha létezik, akkor – egyértelm ˝u.

\subsubsection{teljes rendezett halmaz}

Azt mondjuk, hogy az (R, ≤) rendezett halmaz (a rendezésre nézve) teljes, ha benne
minden nemüres felülr ˝ol korlátos halmaznak létezik pontos fels ˝o korlátja.

\subsubsection{függvény}

Legyenek A, B halmazok, az f ⊂ A × B relációt függvénynek nevezzük, ha minden a ∈ D f esetén az f ({a})
halmaz egyelemű. Ha D f = A, akkor azt mondjuk, hogy f az A-ból B-be képező függvény, és ezt az f : A → B
szimbólummal jelöljük.

\subsubsection{a valós számok axióma-rendszere}

Az R halmazt a valós számok halmazának nevezzük, ha teljesíti az
alábbi axiómákat.
Testaxiómák
Értelmezve van R-ben két művelet, az
f1 : R × R → R, x + y
.= f1(x, y) összeadás és az
f2 : R × R → R, x · y
.= f2(x, y) szorzás,
amelyek kielégítik a következő, úgynevezett testaxiómákat:
1) x + y = y + x , x · y = y · x ∀ x, y ∈ R (kommutativitás),
2) (x + y) + z = x + (y + z), (x · y) · z = x · (y · z)
∀ x, y, z ∈ R (asszociativitás),
3) x · (y + z) = x · y + x · z ∀ x, y, z ∈ R (disztributivitás),
21
22 II. SZÁMOK
4) ∃ 0 ∈ R, hogy x + 0 = x ∀ x ∈ R (∃ zérus, vagy nullelem),
5) ∀ x ∈ R esetén ∃ − x ∈ R, hogy x + (−x) = 0 (∃ additív inverz),
6) ∃ 1 ∈ R, hogy 1 6= 0 és 1 · x = x ∀ x ∈ R (∃ egységelem),
7) ∀ x ∈ R , x 6= 0 esetén ∃ x
−1 ∈ R, hogy x · x
−1 = 1 (∃ multiplikatív
inverz).
Rendezési axiómák
Értelmezve van az R testben egy ≤⊂ R × R rendezési reláció (az I.2.9.
definíció szerinti négy tulajdonsággal), melyekre teljesül még, hogy
(i) ha x, y ∈ R és x ≤ y, akkor x + z ≤ y + z ∀ z ∈ R,
(ii) ha x, y ∈ R, 0 ≤ x és 0 ≤ y, akkor 0 ≤ x · y
(az összeadás és a szorzás monotonitása). Ekkor R-et rendezett testnek
nevezzük.
Teljességi axióma
Az R rendezett test (mint rendezett halmaz) teljes, azaz R bármely nemüres,
felülről korlátos részhalmazának létezik pontos felső korlátja.
Összefoglalva
Az R halmazt a valós számok halmazának nevezzük, ha R teljes rendezett
test.
Megjegyzés. Megmutatható, hogy létezik ilyen halmaz, és bizonyos értelemben
egyértelmű. A lehetséges modellekről később még röviden beszélünk.

\subsubsection{abszolút érték}

Legyen R egy rendezett test, az x ∈ R elem abszolút értékén az
|x| = max {x, −x}
nem negatív számot értjük.

\subsubsection{valós számok távolsága}

Ha x, y ∈ R, akkor a d(x, y)
.= |x − y| számot az x és y
távolságának nevezzük.
Azaz a d : R × R → R függvény távolság (metrika) R-ben

\subsubsection{valós szám nyílt gömbkörnyezete(i)}

Az a ∈ R valós szám r (> 0) sugarú nyílt gömbkörnyezetén
a K(a, r)
.= {x ∈ R | d(x, a) < r} halmazt értjük.
Valójában K(a, r) az a középpontú, 2r hosszúságú nyílt intervallum, azaz
K(a, r) =] a − r, a + r [.

\subsubsection{pozitív valós szám pozitív egész kitevős, egész
kitevős és racionális kitevős hatványai}



\subsubsection{halmaz belső pontja, határpontja, torlódási
pontja}

Legyen adott az E ⊂ R halmaz. Azt mondjuk, hogy
– x ∈ E belső pontja E-nek, ha ∃ K(x, r), hogy K(x, r) ⊂ E;
– x ∈ R külső pontja E-nek, ha belső pontja E komplementerének,
CE-nek
(azaz ∃ K(x, r), K(x, r) ∩ E = ∅);
– x ∈ R határpontja E-nek, ha nem belső és nem külső pontja (azaz
∀ K(x, r)-re K(x, r) ∩ E 6= ∅ ∧ K(x, r) ∩ CE 6= ∅).
E belső pontjainak halmazát E belsejének, a határpontjainak halmazát E
határának nevezzük. E belsejét E◦
jelöli.

Legyen adott az E ⊂ R halmaz. Az x0 ∈ R pontot az E
halmaz torlódási pontjának nevezzük, ha bármely r > 0 esetén a K(x0, r)
környezet tartalmaz x0-tól különböző E-beli pontot, azaz (K(x0, r)\{x0})∩
E 6= ∅.



\subsubsection{nyílt halmaz, zárt halmaz}

Az E ⊂ R halmazt nyíltnak nevezzük, ha minden pontja
belső pont; zártnak nevezzük, ha CE nyílt.

\subsubsection{sorozat monotonitása, konvergenciája}

Az hxni R-beli sorozatot konvergensnek
nevezzük, ha létezik x ∈ R, hogy bármely ε > 0 esetén létezik n(ε) ∈ N,
hogy bármely n ≥ n(ε)-ra (n ∈ N) d(x, xn) < ε teljesül. Az x ∈ R számot
hxni határértékének nevezzük. Azt, hogy hxni konvergens és határértéke x,
így jelöljük: limn→∞
xn = x vagy xn → x.

\subsubsection{sor részlet-összegei, konvergenciája}

1. definíció. Ha adott egy hani R-beli sorozat, akkor azt az hSni sorozatot,
melynél Sn
.=
Pn
k=1
ak végtelen sornak nevezzük és Pan (vagy P∞
n=1
an-nel
jelöljük. Sn-t a sor n-edik részletösszegének, an-t a sor n-edik tagjának
nevezzük. Ha adott még az a0 ∈ R szám is, úgy azt az hSni sorozatot,
melynél Sn =
Pn
k=0
ak is végtelen sornak nevezzük és rá a
P∞
n=0
an jelölést
használjuk.
2. definíció. A
Pan sort konvergensnek mondjuk, ha hSni konvergens,
és a limn→∞
Sn = S számot a sor összegének nevezzük.
Ezen összeget jelölheti a
Pan, illetve a
P∞
n=1
an (ha az összegzés a0-tól indul,
akkor a
P∞
n=0
an) szimbólum is.
A
Pan sor divergens, ha nem konvergens.

\subsubsection{függvény korlátossága, abszolút és helyi maximuma (minimuma)}

2. definíció. Az f : E ⊆ R → R függvény korlátos, ha f(E) korlátos.
Az f : E ⊆ R → R függvény alulról (felülről) korlátos, ha f(E) alulról
(felülről) korlátos.
A sup(f(E)), inf(f(E)) számokat az f pontos felső, illetve pontos alsó korlátjának
(supremumának, illetve infimumának) nevezzük E-n.
3. definíció. Ha az f : E ⊆ R → R függvény esetén létezik x1, x2 ∈ E,
hogy
sup f(E) = f(x1), inf f(E) = f(x2) ,
akkor azt mondjuk, hogy f-nek létezik abszolút maximuma, illetve minimuma
E-n.
Az f : E ⊆ R → R függvénynek az x0 ∈ E-ben helyi (lokális) maximuma,
illetve minimuma van, ha létezik K(x0, δ), hogy x ∈ K(x0, δ)∩E-re
f(x) ≤ f(x0), illetve f(x) ≥ f(x0) teljesül.

\subsubsection{(szigorúan) monoton növekvő (csökkenő) függvény}

Az f : E ⊆ R → R függvény monoton növekvő (csökkenő),
ha ∀ x1, x2 ∈ E, x1 < x2-re f(x1) ≤ f(x2), (illetve f(x1) ≥ f(x2))
teljesül (szigorú monotonitásnál f(x1) < f(x2), illetve f(x1) > f(x2)).
Az f : E ⊆ R → R függvény az x0 ∈ E-n növekvően (csökkenően) halad át,
ha létezik K(x0, δ), hogy ∀ x < x0, x ∈ K(x0, δ) ∩ E esetén
f(x) ≤ f(x0) (f(x) ≥ f(x0))
és x > x0, x ∈ K(x0, δ) ∩ E-re
f(x) ≥ f(x0) (f(x) ≤ f(x0))
teljesül.

\subsubsection{függvény (pontbeli) folytonossága}

Az f : E ⊆ R → R függvény az x0 ∈ E pontban folytonos,
ha ∀ ε > 0-hoz ∃ δ(ε) > 0, hogy ∀ x ∈ E, |x − x0| < δ(ε) esetén |f(x) −
f(x0)| < ε.
Az f : E ⊆ R → R függvény folytonos az A ⊆ E halmazon, ha A minden
pontjában folytonos.

\subsubsection{függvény határértéke}

Az f : E ⊆ R → R függvénynek az x0 ∈ E0 pontban létezik
határértéke, ha létezik A ∈ R, hogy bármely ε > 0 esetén ∃ δ(ε) > 0,
x ∈ E, 0 < |x − x0| < δ(ε) =⇒ |f(x) − A| < ε .
A-t az f függvény x0-beli határértékének nevezzük, és limx→x0
f(x) = A vagy
f(x) → A, ha x → x0, jelöléseket használjuk.

\subsubsection{nevezetes elemi függvények (exp, cos, sin, ch, sh
definíciója hatványsor összegeként, a természetes alapú logaritmus)}
\subsubsection{valós függvények
differenciálhatósága, differenciálhányadosa (deriváltja)}
\subsubsection{további elemi függvények
(expa, loga, tg, ctg, arcsin, arctg, th, arsh, arth)}
\subsubsection{magasabb rendű deriváltak}

Legyen f : ha, bi → R adott függvény. f 0-adik deriváltja:
f
(0) .= f. Ha n ∈ N és f
(n−1) : ha, bi → R értelmezett és differenciálható
függvény, akkor f n-edik deriváltja az f
(n) =

f
(n−1)0
függvény.
Ha ∀ n ∈ N-re ∃ f
(n)
, akkor azt mondjuk, hogy f akárhányszor differenciálható.

%%%%%%%%%%%%%%%%%%%
%%%%ALAPTÉTELEK%%%%
%%%%%%%%%%%%%%%%%%%
 
\subsection{Alaptételek}

\subsubsection{az abszolút érték alapvető tulajdonságai}
\subsubsection{sorozatok és műveletek}
\subsubsection{rendőrtétel}
\subsubsection{a sor konvergenciájának szükséges feltétele}
\subsubsection{az exp, ch, sh, cos és sin függvények tulajdonságai (addíciós tételek, azonosságok, nevezetes határértékek, monoton szakaszok)}
\subsubsection{a differenciálszámítás műveleti szabályai}
\subsubsection{az összetett függvény differenciálhatósága, deriváltja}
\subsubsection{a lokális minimum (maximum) szükséges feltétele}
\subsubsection{a monotonitás (szükséges és) elegendő feltétele(i) differenciálható függvényekre}
 


%%%%%%%%%%%%%%%%%%%
%%%ALAPFELADATOK%%%
%%%%%%%%%%%%%%%%%%%

\subsection{Alapfeladatok}

\subsubsection{az 1/n sorozat és a mértani sorozat határértéke, a sorozatok határértékére vonatkozó műveleti szabályok egyszerűbb alkalmazásai}
\subsubsection{elemi függvények deriváltjai, a differenciálszámítás műveleti szabályainak egyszerűbb alkalmazásai}


\section{További elmélet}

\subsection{Definíciók}

\subsubsection{Halmaz komplementere}
\subsubsection{két halmaz Descartes-szorzata}
\subsubsection{reláció; függvény (vagy reláció) értelmezési tartománya, értékkészlete, inverze}
\subsubsection{halmaz reláció általi képe (vagy halmaz függvény általi képe, ősképe)}
\subsubsection{invertálható függvény}
\subsubsection{természetes, egész és racionális számok}
\subsubsection{megszámlálható számosságú halmaz}
\subsubsection{nyílt, zárt, félig nyílt (zárt) intervallumok}
\subsubsection{kompakt halmaz}
\subsubsection{sorozat fogalma, korlátossága}
\subsubsection{Cauchy-sorozat}
\subsubsection{abszolút (illetve feltételesen) konvergens sor}
\subsubsection{függvény egyoldali folytonossága,
egyoldali határértéke}
\subsubsection{egyenletes folytonosság}
\subsubsection{a végtelen, mint határérték}
\subsubsection{szakadási helyek típusai}
\subsubsection{függvénysorozat és függvénysor pontonkénti illetve egyenletes konvergenciája}
\subsubsection{hatványsor fogalma, konvergencia-sugara}
\subsubsection{páros, páratlan, periodikus
függvény}
\subsubsection{konvex (konkáv) függvény}
\subsubsection{inflexiós hely}


\subsection{Tételek}

\subsubsection{de Morgan azonosságok}
\subsubsection{a távolság (metrika) alapvető tulajdonságai}
\subsubsection{Archimedesi tulajdonság}
\subsubsection{Cantor-féle metszet-tétel}
\subsubsection{n-edik gyök létezése}
\subsubsection{Bolzano–Weierstrass-tétel}
\subsubsection{Heine–Borel-tétel}
\subsubsection{korlátos monoton sorozat konvergenciája}
\subsubsection{sorozatok és rendezés}
\subsubsection{Cauchy-féle konvergencia-kritérium}
\subsubsection{nevezetes sorozatok}
\subsubsection{abszolút konvergens sor konvergenciája}
\subsubsection{két sor összege}
\subsubsection{konvergencia-kritériumok sorokra}
\subsubsection{tizedes törtbe fejtés}
\subsubsection{átviteli elv (függvény folytonosságára, határértékére)}
\subsubsection{folytonosság (illetve határérték) és műveletek, az összetett függvény folytonossága}
\subsubsection{kompakt halmazon folytonos függvény tulajdonságai}
\subsubsection{a határérték és a folytonosság kapcsolata}
\subsubsection{monoton függvények tulajdonságai (invertálhatóság és az inverz-függvény monotonitása, folytonossága, egyoldali határértékek, szakadási helyek számossága)}
\subsubsection{Weierstrass elegendő feltétele függvénysorok egyenletes konvergenciájára}
\subsubsection{az összegfüggvény folytonossága}
\subsubsection{Cauchy–Hadamard-tétel}
\subsubsection{differenciálhatóság és folytonosság kapcsolata}
\subsubsection{az inverz függvény differenciálhatósága, deriváltja}
\subsubsection{hatványsorok differenciálhatósága}
\subsubsection{deriválási szabályok magasabb rendű deriváltakra}
\subsubsection{Leibniz-szabály}
\subsubsection{középérték-tételek (Cauchy-, Lagrange-, Rolle-)}
\subsubsection{Taylor tétele}
\subsubsection{a
szélsőérték elegendő feltétele}
\subsubsection{a konvexitás (konkavitás) elegendő feltétele (kétszer)
differenciálható függvényekre}
\subsubsection{L’Hospital-szabály}


\section{Feladat-típusok}

\subsubsection{halmaz-műveletek azonosságainak igazolása}
\subsubsection{műveletek konkrét halmazokkal}
\subsubsection{konkrét halmaz konkrét függvény általi képe, ősképe}
\subsubsection{halmaz belső (illetve határ-, torlódási) pontjainak meghatározása}
\subsubsection{sorozatok konvergenciájának vizsgálata, a határérték meghatározása}
\subsubsection{sorok konvergenciájának vizsgálata; mértani (és egyéb speciális) sorok (illetve
ilyenek lineáris kombinációi) összegének meghatározása}
\subsubsection{függvények határértékének meghatározása algebrai átalakítások segítségével}
\subsubsection{hatványsorok konvergencia-sugarának meghatározása}
\subsubsection{hatványsorok (és ezekre
visszavezethető függvénysorok) konvergencia-tartománya}
\subsubsection{elemi függvények deriválása}
\subsubsection{szorzatok magasabb rendű deriváltjai}



\end{document}
