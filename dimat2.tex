\documentclass[12pt]{article}

\usepackage[utf8]{inputenc}
\usepackage{xcolor}   %May be necessary if you want to color links
\usepackage{hyperref}
\usepackage{amsmath}
\hypersetup{
    colorlinks=false, %set true if you want colored links
    linkbordercolor = {white}
}
 
\title{\Huge Dimat 2 \\
\LARGE University of Debrecen \\
\large DE-IK P.T.I.}
\author{Peter Panko}

 
\begin{document}

\clearpage\maketitle
\thispagestyle{empty}
\newpage


\tableofcontents 

\newpage



%%%%%%%%%%%%%%%%%%
%%%%%%BEUGRÓ%%%%%%
%%%%%%%%%%%%%%%%%%

\section{Beugró}

%%%%%%%%%%%%%%%%%%%
%%%%%ALAPDEFEK%%%%%
%%%%%%%%%%%%%%%%%%%

\subsection{Alapdefiníciók}

\subsubsection{Halmaz unió}

\begin{center}
$A \cup B = \{x \in X\ |\ x \in A\ \text{vagy}\ x \in B\}$
\end{center}
halmazt értjük.

\subsubsection{Halmaz metszet}

Legyen $X$ egy nemüres halmaz, $A, B \subset X$. Ekkor az $A$ és $B$ halmazok metszetén az
\begin{center}
$A \cap B = \{x \in X\ |\ x \in A\ \text{és}\ x \in B\}$
\end{center}
halmazt értjük.


@

%%%%%%%%%%%%%%%%%%%
%%%%ALAPTÉTELEK%%%%
%%%%%%%%%%%%%%%%%%%
 
\subsection{Alaptételek}

\subsubsection{az abszolút érték alapvető tulajdonságai}
\subsubsection{sorozatok és műveletek}
\subsubsection{rendőrtétel}
\subsubsection{a sor konvergenciájának szükséges feltétele}
\subsubsection{az exp, ch, sh, cos és sin függvények tulajdonságai (addíciós tételek, azonosságok, nevezetes határértékek, monoton szakaszok)}
\subsubsection{a differenciálszámítás műveleti szabályai}
\subsubsection{az összetett függvény differenciálhatósága, deriváltja}
\subsubsection{a lokális minimum (maximum) szükséges feltétele}
\subsubsection{a monotonitás (szükséges és) elegendő feltétele(i) differenciálható függvényekre}
 


%%%%%%%%%%%%%%%%%%%
%%%ALAPFELADATOK%%%
%%%%%%%%%%%%%%%%%%%

\subsection{Alapfeladatok}

\subsubsection{az 1/n sorozat és a mértani sorozat határértéke, a sorozatok határértékére vonatkozó műveleti szabályok egyszerűbb alkalmazásai}
\subsubsection{elemi függvények deriváltjai, a differenciálszámítás műveleti szabályainak egyszerűbb alkalmazásai}


\section{További elmélet}

\subsection{Definíciók}

\subsubsection{Halmaz komplementere}
\subsubsection{két halmaz Descartes-szorzata}
\subsubsection{reláció; függvény (vagy reláció) értelmezési tartománya, értékkészlete, inverze}
\subsubsection{halmaz reláció általi képe (vagy halmaz függvény általi képe, ősképe)}
\subsubsection{invertálható függvény}
\subsubsection{természetes, egész és racionális számok}
\subsubsection{megszámlálható számosságú halmaz}
\subsubsection{nyílt, zárt, félig nyílt (zárt) intervallumok}
\subsubsection{kompakt halmaz}
\subsubsection{sorozat fogalma, korlátossága}
\subsubsection{Cauchy-sorozat}
\subsubsection{abszolút (illetve feltételesen) konvergens sor}
\subsubsection{függvény egyoldali folytonossága,
egyoldali határértéke}
\subsubsection{egyenletes folytonosság}
\subsubsection{a végtelen, mint határérték}
\subsubsection{szakadási helyek típusai}
\subsubsection{függvénysorozat és függvénysor pontonkénti illetve egyenletes konvergenciája}
\subsubsection{hatványsor fogalma, konvergencia-sugara}
\subsubsection{páros, páratlan, periodikus
függvény}
\subsubsection{konvex (konkáv) függvény}
\subsubsection{inflexiós hely}


\subsection{Tételek}

\subsubsection{de Morgan azonosságok}
\subsubsection{a távolság (metrika) alapvető tulajdonságai}
\subsubsection{Archimedesi tulajdonság}
\subsubsection{Cantor-féle metszet-tétel}
\subsubsection{n-edik gyök létezése}
\subsubsection{Bolzano–Weierstrass-tétel}
\subsubsection{Heine–Borel-tétel}
\subsubsection{korlátos monoton sorozat konvergenciája}
\subsubsection{sorozatok és rendezés}
\subsubsection{Cauchy-féle konvergencia-kritérium}
\subsubsection{nevezetes sorozatok}
\subsubsection{abszolút konvergens sor konvergenciája}
\subsubsection{két sor összege}
\subsubsection{konvergencia-kritériumok sorokra}
\subsubsection{tizedes törtbe fejtés}
\subsubsection{átviteli elv (függvény folytonosságára, határértékére)}
\subsubsection{folytonosság (illetve határérték) és műveletek, az összetett függvény folytonossága}
\subsubsection{kompakt halmazon folytonos függvény tulajdonságai}
\subsubsection{a határérték és a folytonosság kapcsolata}
\subsubsection{monoton függvények tulajdonságai (invertálhatóság és az inverz-függvény monotonitása, folytonossága, egyoldali határértékek, szakadási helyek számossága)}
\subsubsection{Weierstrass elegendő feltétele függvénysorok egyenletes konvergenciájára}
\subsubsection{az összegfüggvény folytonossága}
\subsubsection{Cauchy–Hadamard-tétel}
\subsubsection{differenciálhatóság és folytonosság kapcsolata}
\subsubsection{az inverz függvény differenciálhatósága, deriváltja}
\subsubsection{hatványsorok differenciálhatósága}
\subsubsection{deriválási szabályok magasabb rendű deriváltakra}
\subsubsection{Leibniz-szabály}
\subsubsection{középérték-tételek (Cauchy-, Lagrange-, Rolle-)}
\subsubsection{Taylor tétele}
\subsubsection{a
szélsőérték elegendő feltétele}
\subsubsection{a konvexitás (konkavitás) elegendő feltétele (kétszer)
differenciálható függvényekre}
\subsubsection{L’Hospital-szabály}


\section{Feladat-típusok}

\subsubsection{halmaz-műveletek azonosságainak igazolása}
\subsubsection{műveletek konkrét halmazokkal}
\subsubsection{konkrét halmaz konkrét függvény általi képe, ősképe}
\subsubsection{halmaz belső (illetve határ-, torlódási) pontjainak meghatározása}
\subsubsection{sorozatok konvergenciájának vizsgálata, a határérték meghatározása}
\subsubsection{sorok konvergenciájának vizsgálata; mértani (és egyéb speciális) sorok (illetve
ilyenek lineáris kombinációi) összegének meghatározása}
\subsubsection{függvények határértékének meghatározása algebrai átalakítások segítségével}
\subsubsection{hatványsorok konvergencia-sugarának meghatározása}
\subsubsection{hatványsorok (és ezekre
visszavezethető függvénysorok) konvergencia-tartománya}
\subsubsection{elemi függvények deriválása}
\subsubsection{szorzatok magasabb rendű deriváltjai}



\end{document}
