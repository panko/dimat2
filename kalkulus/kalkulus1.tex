\documentclass{article}
\usepackage[utf8]{inputenc}
\usepackage{xcolor}   %May be necessary if you want to color links
\usepackage{hyperref}
\hypersetup{
    colorlinks=false, %set true if you want colored links
    linkbordercolor = {white}
}
 
\title{\Huge Kalkulus 1 \\
\LARGE University of Debrecen \\
\large DE-IK P.T.I.}
\author{Peter Panko}
\date{ }
 
\begin{document}

\clearpage\maketitle
\thispagestyle{empty}

\newpage
\tableofcontents
\newpage


\section{Beugró}



\subsection{Alapdefiníciók}

\subsubsection{First subsubsection}
 
halmaz-műveletek (unió, metszet, különbség); rendezési reláció;
felülről (alulról) korlátos halmaz; pontos felső (alsó) korlát; teljes rendezett halmaz;
függvény; a valós számok axióma-rendszere; abszolút érték; valós számok távolsága;
valós szám nyílt gömbkörnyezete(i); pozitív valós szám pozitív egész kitevős, egész
kitevős és racionális kitevős hatványai; halmaz belső pontja, határpontja, torlódási
pontja; nyílt halmaz, zárt halmaz; sorozat monotonitása, konvergenciája; sor részlet-
összegei, konvergenciája; függvény korlátossága, abszolút és helyi maximuma
(minimuma); (szigorúan) monoton növekvő (csökkenő) függvény; függvény (pontbeli)
folytonossága; függvény határértéke; nevezetes elemi függvények (exp, cos, sin, ch, sh
definíciója hatványsor összegeként, a természetes alapú logaritmus); valós függvények
differenciálhatósága, differenciálhányadosa (deriváltja); további elemi függvények
(expa, loga, tg, ctg, arcsin, arctg, th, arsh, arth); magasabb rendű deriváltak.
 

 
\subsection{Alaptételek}
 
az abszolút érték alapvető tulajdonságai; sorozatok és műveletek; rendőrtétel;
a sor konvergenciájának szükséges feltétele; az exp, ch, sh, cos és sin függvények
tulajdonságai (addíciós tételek, azonosságok, nevezetes határértékek, monoton
szakaszok); a differenciálszámítás műveleti szabályai; az összetett függvény
differenciálhatósága, deriváltja; a lokális minimum (maximum) szükséges feltétele; a
monotonitás (szükséges és) elegendő feltétele(i) differenciálható függvényekre.

\subsection{Alapfeladatok}

az 1/n sorozat és a mértani sorozat határértéke, a sorozatok határértékére
vonatkozó műveleti szabályok egyszerűbb alkalmazásai; elemi függvények deriváltjai,
a differenciálszámítás műveleti szabályainak egyszerűbb alkalmazásai. 


\section{További elmélet}

\subsection{Definíciók}
 
halmaz komplementere; két halmaz Descartes-szorzata; reláció; függvény
(vagy reláció) értelmezési tartománya, értékkészlete, inverze; halmaz reláció általi képe
(vagy halmaz függvény általi képe, ősképe); invertálható függvény; természetes, egész
és racionális számok; megszámlálható számosságú halmaz; nyílt, zárt, félig nyílt (zárt)
intervallumok; kompakt halmaz; sorozat fogalma, korlátossága; Cauchy-sorozat;
abszolút (illetve feltételesen) konvergens sor; függvény egyoldali folytonossága,
egyoldali határértéke; egyenletes folytonosság; a végtelen, mint határérték; szakadási
helyek típusai; függvénysorozat és függvénysor pontonkénti illetve egyenletes
konvergenciája; hatványsor fogalma, konvergencia-sugara; páros, páratlan, periodikus
függvény; konvex (konkáv) függvény; inflexiós hely.

\subsection{Tételek}


de Morgan azonosságok; a távolság (metrika) alapvető tulajdonságai;
Archimedesi tulajdonság; Cantor-féle metszet-tétel; n-edik gyök létezése; Bolzano–
Weierstrass-tétel; Heine–Borel-tétel; korlátos monoton sorozat konvergenciája;
sorozatok és rendezés; Cauchy-féle konvergencia-kritérium; nevezetes sorozatok;
abszolút konvergens sor konvergenciája; két sor összege; konvergencia-kritériumok
sorokra; tizedes törtbe fejtés; átviteli elv (függvény folytonosságára, határértékére);
folytonosság (illetve határérték) és műveletek, az összetett függvény folytonossága;
kompakt halmazon folytonos függvény tulajdonságai; a határérték és a folytonosság
kapcsolata; monoton függvények tulajdonságai (invertálhatóság és az inverz-függvény
monotonitása, folytonossága, egyoldali határértékek, szakadási helyek számossága);
Weierstrass elegendő feltétele függvénysorok egyenletes konvergenciájára; az
összegfüggvény folytonossága; Cauchy–Hadamard-tétel; differenciálhatóság és
folytonosság kapcsolata; az inverz függvény differenciálhatósága, deriváltja;
hatványsorok differenciálhatósága; deriválási szabályok magasabb rendű deriváltakra,
Leibniz-szabály; középérték-tételek (Cauchy-, Lagrange-, Rolle-); Taylor tétele; a
szélsőérték elegendő feltétele; a konvexitás (konkavitás) elegendő feltétele (kétszer)
differenciálható függvényekre; L’Hospital-szabály. 

\section{Feladat-típusok}

halmaz-műveletek azonosságainak igazolása;
 műveletek konkrét halmazokkal;
 konkrét halmaz konkrét függvény általi képe, ősképe;
 halmaz belső (illetve határ-, torlódási) pontjainak meghatározása;
 sorozatok konvergenciájának vizsgálata, a határérték meghatározása;
 sorok konvergenciájának vizsgálata; mértani (és egyéb speciális) sorok (illetve
ilyenek lineáris kombinációi) összegének meghatározása;
 függvények határértékének meghatározása algebrai átalakítások segítségével;
 hatványsorok konvergencia-sugarának meghatározása; hatványsorok (és ezekre
visszavezethető függvénysorok) konvergencia-tartománya;
 elemi függvények deriválása; szorzatok magasabb rendű deriváltjai;

\end{document}